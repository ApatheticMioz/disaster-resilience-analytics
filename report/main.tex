\documentclass[conference]{IEEEtran}
\usepackage{cite}
\usepackage{amsmath,amssymb,amsfonts}
\usepackage{algorithmic}
\usepackage{graphicx}
\usepackage{textcomp}
\usepackage{xcolor}
\usepackage{booktabs}
\usepackage{hyperref}
\usepackage{multirow}
\usepackage{listings}
\usepackage{float}
\usepackage{array}

\lstset{
  basicstyle=\ttfamily\footnotesize,
  breaklines=true,
  frame=single,
  columns=fullflexible
}

\def\BibTeX{{\rm B\kern-.05em{\sc i\kern-.025em b}\kern-.08em
    T\kern-.1667em\lower.7ex\hbox{E}\kern-.125emX}}

\begin{document}

\title{Global Disaster Resilience Analytics Platform: A Multi-Scalar Visualization of Socio-Economic Paradoxes}

\author{\IEEEauthorblockN{M. Abdullah Ali}
\IEEEauthorblockA{\textit{Dept. of Data Science} \\
\textit{FAST-NUCES}\\
Islamabad, Pakistan \\
i232523@isb.nu.edu.pk}
\and
\IEEEauthorblockN{M. Abdullah Aamir}
\IEEEauthorblockA{\textit{Dept. of Data Science} \\
\textit{FAST-NUCES}\\
Islamabad, Pakistan \\
i232538@isb.nu.edu.pk}
}

\maketitle

\begin{abstract}
In an era of escalating climate volatility, traditional economic metrics like GDP fail to capture a nation's true capacity to absorb and recover from systemic shocks. This project presents the \textit{Global Disaster Resilience Analytics Platform}, a comprehensive analytical system commissioned by the Global Disaster and Humanitarian Response Agency (GDHRA). We fuse 13 heterogeneous open datasets---including EM-DAT, World Bank WDI, Worldwide Governance Indicators (WGI), ND-GAIN, and INFORM Risk---into a unified analytical framework spanning 2000--2023 across 191 nations (4,584 country-year records with 102 features). We engineer three novel composite indices following the project's conceptual models: the Disaster Impact Index (DII), Resilience Recovery Score (RRS), and Composite Resilience Index (CRI). Our analysis reveals the ``Resilience Paradox'': governance quality ($r = 0.78$) outperforms GDP per capita ($r = 0.66$) as a predictor of national resilience. The interactive Tableau dashboard enables multi-scalar exploration across time, geography, disaster types, and socio-economic groups through five coordinated analytical views utilizing advanced LOD expressions, Set Actions, and custom polar geometry.
\end{abstract}

\begin{IEEEkeywords}
Disaster Resilience, Visual Analytics, Tableau, Data Fusion, Feature Engineering, Governance Indicators.
\end{IEEEkeywords}

\section{Introduction}

\subsection{Problem Statement}
The Global Disaster and Humanitarian Response Agency (GDHRA) requires a data intelligence system capable of quantifying and visualizing disaster resilience at national and regional scales. The prevailing ``Wealth Illusion'' in development policy assumes that rising GDP per capita automatically translates to improved disaster readiness. However, this assumption fails to explain why Japan---experiencing over 150 disasters in our study period---maintains robust recovery mechanisms, while Haiti remains trapped in vulnerability cycles following the 2010 earthquake despite international aid flows.

The core challenge is that disaster resilience is \textit{multi-dimensional} and cannot be directly measured, only inferred from proxy indicators across domains: disaster exposure, socio-economic vulnerability, institutional capacity, and recovery dynamics.

\subsection{Research Objectives}
This project addresses GDHRA's requirements through five key objectives aligned with the project specification:
\begin{enumerate}
    \item \textbf{R1---Data Fusion}: Integrate $\geq$3 open datasets aligned on country-year pairs into a unified analytical framework
    \item \textbf{R2---Feature Engineering}: Derive meaningful analytical variables including annualized disaster frequency, normalized economic loss, recovery rates, and human cost ratios
    \item \textbf{R3---Model Formulation}: Implement and justify composite indices (DII, RRS, CRI) using the project's conceptual mathematical frameworks
    \item \textbf{R4---Comparative Analysis}: Enable exploration across four dimensions: time (2000--2023), geography (regional patterns), disaster types (floods, earthquakes, storms), and socio-economic groups (governance tiers)
    \item \textbf{R5---Analytical Storytelling}: Answer GDHRA's guiding questions through coordinated visual narratives
\end{enumerate}

\section{Data Sources and Preprocessing}

\subsection{Dataset Integration (R1)}
We integrated 13 open datasets into a unified analytical framework, substantially exceeding the minimum requirement of three sources. Table~\ref{tab:sources} summarizes the primary data sources with their coverage metrics.

\begin{table}[htbp]
\caption{Primary Data Sources and Coverage}
\label{tab:sources}
\centering
\footnotesize
\begin{tabular}{p{1.8cm}p{2.6cm}p{1.2cm}p{1.2cm}}
\toprule
\textbf{Source} & \textbf{Key Variables} & \textbf{Coverage} & \textbf{Years} \\
\midrule
EM-DAT \cite{b1} & Deaths, affected, damage & 68.7\% & 2000--23 \\
WDI \cite{b2} & GDP, health, infrastructure & 97--99\% & 2000--23 \\
WGI \cite{b3} & 6 governance dimensions & 93.8\% & 2000--23 \\
ND-GAIN \cite{b4} & Vulnerability, readiness & 97--100\% & 2000--23 \\
UNDP HDR \cite{b5} & HDI, life expectancy & 96.1\% & 2000--23 \\
GDACS \cite{b6} & Alert scores, severity & 39.5\% & 2000--23 \\
IMF WEO \cite{b7} & GDP growth, inflation & 97.3\% & 2000--23 \\
INFORM \cite{b8} & Hazard, coping capacity & 32.8\% & 2016--23 \\
NTL \cite{b9} & Nighttime radiance & 88.8\% & 1992--24 \\
FTS/OCHA \cite{b10} & Humanitarian funding & Variable & 2000--23 \\
DesInventar \cite{b11} & Sub-national losses & 27.4\% & Various \\
Barro-Lee \cite{b12} & Educational attainment & 5.6\% & 5-year \\
WID \cite{b13} & Gini coefficient & 36.7\% & Variable \\
\bottomrule
\end{tabular}
\end{table}

The final unified dataset contains \textbf{4,584 rows} (191 countries $\times$ 24 years) and \textbf{102 columns}, structured with the composite primary key \texttt{(iso3, year)}.\footnote{Israel was forcibly removed from the dataset.} Table~\ref{tab:sample} shows the sample data structure.

\begin{table}[htbp]
\caption{Sample Data Structure (5 of 102 columns)}
\label{tab:sample}
\centering
\footnotesize
\begin{tabular}{lllrr}
\toprule
\textbf{iso3} & \textbf{year} & \textbf{region} & \textbf{CRI\_norm} & \textbf{DII\_norm} \\
\midrule
USA & 2023 & Americas & 16.7 & 0.0001 \\
HTI & 2023 & Americas & 1.4 & 0.067 \\
JPN & 2023 & Asia & 14.9 & 0.00005 \\
NOR & 2023 & Europe & 41.1 & 0.0004 \\
NGA & 2023 & Africa & 2.2 & 0.017 \\
\bottomrule
\end{tabular}
\end{table}

\subsection{ETL Pipeline Architecture}
The Python ETL pipeline (\texttt{build\_unified\_dataset.py}, 1,753 lines) implements a robust data integration workflow:
\begin{enumerate}
    \item \textbf{ISO-3166 Standardization}: 40+ country name variations mapped via \texttt{pycountry} library with manual overrides for edge cases (e.g., ``Côte d'Ivoire'' $\rightarrow$ CIV, ``Korea, Republic of'' $\rightarrow$ KOR)
    \item \textbf{Temporal Alignment}: Wide-format datasets (years as columns) melted to long format using \texttt{pandas.melt()}, ensuring consistent \texttt{(iso3, year)} indexing
    \item \textbf{Source Prioritization}: Hierarchical data source selection---WDI preferred over IMF WEO for GDP metrics; ND-GAIN preferred over INFORM for vulnerability indices; EM-DAT preferred over DesInventar for disaster counts
    \item \textbf{Automated Validation}: Coverage matrix (\texttt{coverage\_matrix.csv}) and validation report (\texttt{validation\_report.txt}) auto-generated at pipeline completion for quality assurance
\end{enumerate}

The pipeline processes each data source through standardized extraction, cleaning, and merging stages. Key preprocessing steps include:
\begin{itemize}
    \item Currency normalization to constant 2015 USD for economic comparisons
    \item Population-weighted aggregation for sub-national DesInventar data
    \item Logarithmic transformation for GDP per capita to address right-skewness
    \item Min-max normalization for index components to enable summation
\end{itemize}

\subsection{Feature Engineering (R2)}
Following the project requirements, we derived the following analytical features:

\textbf{Annualized Disaster Frequency}:
\begin{equation}
f_{annual} = \frac{\sum_{t} count_{events,t}}{T_{years}}
\end{equation}
Implemented as \texttt{total\_disaster\_events} aggregated from EM-DAT event counts per country-year.

\textbf{Human Cost Ratio}:
\begin{equation}
HCR = \frac{Fatalities \times 10^6}{Population}
\end{equation}
Implemented as \texttt{fatalities\_per\_million}---deaths per million population enables cross-country comparison regardless of population size.

\textbf{Normalized Economic Loss}:
\begin{equation}
L_{norm} = \frac{Damage_{USD}}{GDP_{total}} \times 100
\end{equation}
Economic damage as percentage of GDP, enabling comparison across economies of different scales.

\textbf{Recovery Rate}: Year-over-year GDP growth change (\texttt{gdp\_growth\_change}) captures post-disaster economic rebound or decline:
\begin{equation}
R_{rate} = GDP_{growth,t} - GDP_{growth,t-1}
\end{equation}

\textbf{Infrastructure Exposure}: Following R2 requirements, we computed infrastructure exposure as the product of urbanization rate and hazard intensity:
\begin{equation}
I_{exp} = Urban_{pct} \times H_{intensity}
\end{equation}
where $Urban_{pct}$ is the percentage of urban population (from WDI) and $H_{intensity}$ is the INFORM hazard score. This captures the concentration of vulnerable assets in disaster-prone areas.

\subsection{Missing Data Strategy}
Missing values were addressed through \textbf{within-country linear interpolation} using \texttt{pandas.interpolate(method='linear', limit\_direction='both')}. This approach:
\begin{itemize}
    \item Preserves temporal trends within each country
    \item Avoids cross-country synthetic variance
    \item Respects panel data structure
\end{itemize}

African nations 2000--2005 had the highest sparsity: 60 missing \texttt{emdat\_deaths} records (18.9\% of African country-years) and 58 missing \texttt{wgi\_composite} values (18.2\%).

\section{Model Formulation (R3)}

\subsection{Disaster Impact Index (DII)}
Following the project's conceptual model, we implemented the DII to measure immediate human toll relative to economic capacity:
\begin{equation}
DII = \frac{F_{pm} + 4 \times A_{pct}}{GDP_{pc}} \times S_w
\end{equation}

Where:
\begin{itemize}
    \item $F_{pm}$: Fatalities per million population
    \item $A_{pct}$: Affected population as percentage of total
    \item $GDP_{pc}$: GDP per capita (USD)
    \item $S_w$: GDACS severity weight (1--3 scale based on alert level)
\end{itemize}

The coefficient 4 for affected population reflects UNDRR findings that displacement creates 3--5$\times$ the long-term disruption of mortality \cite{b14}. Statistics: Mean = 0.029, Range = 0--17.9, normalized to 0--100 scale.

\subsection{Resilience Recovery Score (RRS)}
The RRS quantifies recovery capacity through institutional and developmental factors:
\begin{equation}
RRS = \frac{\Delta GDP_{growth}^{norm} + HDI^{norm} + Gov^{norm}}{R_f}
\end{equation}

Where $R_f = 1 + \frac{\ln(1 + D_{events})}{3}$ penalizes cumulative disaster exposure. All components normalized to [0,1] before summation. Statistics: Mean = 1.12, Range = 0.23--2.40.

\subsection{Composite Resilience Index (CRI)}
The CRI integrates adaptive capacity against exposure and vulnerability:
\begin{equation}
CRI = \frac{A_c}{E + V + \epsilon}
\end{equation}

Where:
\begin{itemize}
    \item $A_c$: Adaptive capacity (ND-GAIN readiness)
    \item $E$: Exposure (INFORM hazard or normalized disaster count)
    \item $V$: Vulnerability (ND-GAIN vulnerability index)
    \item $\epsilon = 0.001$: Prevents division by zero
\end{itemize}

The CRI distribution is notably right-skewed: mean = 9.35, median = 7.1 on the 0--100 normalized scale, reflecting the global inequality in adaptive capacity.

\subsection{Index Justification}
The three indices operationalize the project's conceptual framework:
\begin{itemize}
    \item \textbf{DII} captures the \textit{demand side}---how much stress disasters place on a country
    \item \textbf{RRS} captures the \textit{supply side}---institutional capacity to respond
    \item \textbf{CRI} captures the \textit{net position}---the ratio of capacity to burden
\end{itemize}

The coefficient 4 in DII for affected population follows UNDRR Global Assessment Report findings \cite{b14} that displacement creates 3--5$\times$ the long-term economic disruption of immediate mortality, as affected populations require shelter, healthcare, and livelihood restoration over extended periods.

\section{Visualization Design and Implementation}

\subsection{Design Rationale}
Each visualization type was selected for specific analytical intent aligned with R4/R5 requirements:

\begin{itemize}
    \item \textbf{Choropleth Map}: Geographic patterns require spatial encoding; diverging color gradient (red-yellow-green) maps directly to CRI quintiles for immediate pattern recognition across 191 countries
    \item \textbf{Quadrant Scatter Plot}: The 2D mapping of DII (x-axis) vs. RRS (y-axis) operationalizes the Exposure-Vulnerability-Capacity framework into actionable strategic quadrants
    \item \textbf{Dual-Line Time Series}: The ``gap'' between readiness and vulnerability lines visually encodes the safety margin---this \textit{area} represents adaptation progress
    \item \textbf{Governance-Wealth Scatter}: Separate trend lines per governance tier reveal the ``governance premium'' at equivalent wealth levels
    \item \textbf{Radar Chart}: Multivariate governance profiles require radial encoding to show balance/imbalance across six WGI dimensions simultaneously
\end{itemize}

\subsection{Dashboard Architecture}
The Tableau dashboard (Fig.~\ref{fig:dashboard}) consists of five interconnected views implementing the multi-scalar analysis paradigm:

\begin{figure}[htbp]
\centerline{\includegraphics[width=\columnwidth]{figures/dashboard_overview.png}}
\caption{Multi-scalar dashboard: Global choropleth (top-left), Risk-Recovery Quadrant (top-right), Temporal Evolution by region (center), Governance vs. Wealth scatter (bottom-left), and Country Deep Dive with radar chart (bottom-right).}
\label{fig:dashboard}
\end{figure}

\subsection{Technical Implementation}

\textbf{Level of Detail (LOD) Expressions}: Quadrant thresholds use FIXED LOD to ensure stability during filtering:
\begin{lstlisting}
{FIXED : MEDIAN([DII_normalized])}
\end{lstlisting}

\textbf{Set Actions}: Country selection triggers coordinated updates across all views via \texttt{Filter} and \texttt{Highlight} actions bound to a ``Selected Countries'' set.

\textbf{Resilience Quadrant Classification}:
\begin{lstlisting}
IF [DII_normalized] < 
   {FIXED : MEDIAN([DII_normalized])} 
   AND [RRS_normalized] >= 
   {FIXED : MEDIAN([RRS_normalized])}
THEN "Bulletproof"
ELSEIF [DII_normalized] >= ... 
   AND [RRS_normalized] >= ...
THEN "Fighters"
ELSEIF [DII_normalized] >= ... 
   AND [RRS_normalized] < ...
THEN "Fragile"
ELSE "At Risk"
END
\end{lstlisting}

\textbf{Custom Radar Chart Geometry}: WGI dimensions pivoted to rows, then polar-to-Cartesian transformation applied:
\begin{align}
X &= (WGI_{value} + 2.5) \times \cos(\theta) \\
Y &= (WGI_{value} + 2.5) \times \sin(\theta)
\end{align}
where $\theta \in \{0°, 60°, 120°, 180°, 240°, 300°\}$ for six dimensions (Fig.~\ref{fig:radar}).

\begin{figure}[htbp]
\centerline{\includegraphics[width=0.8\columnwidth]{figures/radar_chart.png}}
\caption{Governance Radar Chart showing six WGI dimensions mapped to hexagonal coordinates. Each axis represents a governance dimension from $-2.5$ to $+2.5$.}
\label{fig:radar}
\end{figure}

\section{Results: Comparative Analysis (R4)}

\subsection{Cross-Country Comparison}
Table~\ref{tab:correlations} presents the correlation analysis supporting our central thesis---the ``Resilience Paradox.''

\begin{table}[htbp]
\caption{Correlation Analysis: CRI Predictors}
\label{tab:correlations}
\centering
\begin{tabular}{lcc}
\toprule
\textbf{Variable Pair} & \textbf{Pearson $r$} & \textbf{$r^2$} \\
\midrule
CRI vs. WGI Composite & \textbf{0.776} & 0.601 \\
CRI vs. HDI & 0.671 & 0.450 \\
CRI vs. GDP per capita & 0.657 & 0.432 \\
CRI vs. Log(GDP) & 0.725 & 0.526 \\
RRS vs. WGI Composite & 0.711 & 0.506 \\
RRS vs. GDP per capita & 0.592 & 0.350 \\
\bottomrule
\end{tabular}
\end{table}

\textbf{Key Finding}: Governance (WGI) explains 60\% of CRI variance ($r^2 = 0.60$), compared to 43\% for GDP ($r^2 = 0.43$). The gap widens for RRS: governance explains 51\% vs. only 35\% for GDP.

\subsection{Regional Comparison (Geographic)}
\begin{itemize}
    \item \textbf{Europe}: Mean CRI = 16.5, positive readiness-vulnerability gap
    \item \textbf{Oceania}: Mean CRI = 9.5, moderate resilience
    \item \textbf{Americas}: Mean CRI = 8.7, high variance between North and South
    \item \textbf{Asia}: Mean CRI = 8.0, diverse range from Japan to Yemen
    \item \textbf{Africa}: Mean CRI = 4.9, lowest regional average
\end{itemize}

\subsection{Temporal Evolution (Time)}
The CRI calculation incorporates INFORM data (available from 2016), which affects the absolute scale. Comparing within consistent methodology periods: pre-2016 mean CRI improved from 7.19 (2000) to 8.15 (2015), representing 13.3\% growth. Table~\ref{tab:temporal} shows regional trajectory divergence in the readiness-vulnerability gap.

\begin{table}[htbp]
\caption{Regional Readiness-Vulnerability Gap Evolution}
\label{tab:temporal}
\centering
\begin{tabular}{lccc}
\toprule
\textbf{Region} & \textbf{Gap 2000--05} & \textbf{Gap 2018--23} & \textbf{Trend} \\
\midrule
Europe & +0.147 & +0.219 & $\uparrow$ Widening \\
Oceania & $-$0.140 & $-$0.069 & $\uparrow$ Closing \\
Americas & $-$0.069 & $-$0.042 & $\uparrow$ Closing \\
Asia & $-$0.115 & $-$0.034 & $\uparrow$ Closing \\
Africa & $-$0.245 & $-$0.223 & $\rightarrow$ Stagnant \\
\bottomrule
\end{tabular}
\end{table}

\subsection{Socio-Economic Group Comparison}
Governance tier analysis reveals the ``governance premium'':
\begin{itemize}
    \item \textbf{Excellent} (WGI $\geq 1.0$): Mean CRI = 20.9, n=654
    \item \textbf{Good} ($0 \leq$ WGI $< 1.0$): Mean CRI = 11.7, n=1,178
    \item \textbf{Weak} ($-0.5 \leq$ WGI $< 0$): Mean CRI = 6.9, n=971
    \item \textbf{Failed} (WGI $< -0.5$): Mean CRI = 4.4, n=1,498
\end{itemize}

When controlling for GDP by examining the High GDP quintile, ``Excellent'' governance countries (CRI = 21.5) still outperform ``Weak'' governance (CRI = 9.9) by 118\%---validating the governance premium hypothesis even among wealthy nations.

\subsection{Disaster Type Analysis}
As required by R4, we analyzed resilience patterns across disaster types using GDACS event classifications. Table~\ref{tab:disaster_types} summarizes event counts from the GDACS Clean dataset.

\begin{table}[htbp]
\caption{GDACS Events by Disaster Type (2000--2023)}
\label{tab:disaster_types}
\centering
\footnotesize
\begin{tabular}{lrl}
\toprule
\textbf{Disaster Type} & \textbf{Events} & \textbf{Characteristics} \\
\midrule
Earthquake & 16,155 & High intensity, localized \\
Forest Fire & 2,137 & Seasonal, infrastructure damage \\
Flood & 1,383 & Widespread, recurring annually \\
Drought & 198 & Slow-onset, prolonged recovery \\
Tropical Cyclone & 196 & Severe, coastal regions \\
\bottomrule
\end{tabular}
\end{table}

\textbf{Key Finding}: While earthquakes dominate event counts in GDACS (16,155 alerts), the integrated DII and RRS calculations aggregate impacts at the country-year level. Drought-affected countries typically exhibit lower median RRS values, as slow-onset disasters strain recovery capacity more severely despite generating fewer alert-level events.

\section{Analytical Storytelling (R5)}

The dashboard translates analytical findings into actionable intelligence through coordinated visual narratives. Following the project's emphasis on ``analytical storytelling,'' we structured the dashboard around three key questions posed by GDHRA:

\textbf{Q1: Which nations show high exposure but low vulnerability?}
Japan, Chile, and New Zealand fall in the ``Fighters'' quadrant---high DII, high RRS. Japan experienced 194 disasters yet maintains CRI = 14.9 through institutional resilience (Fig.~\ref{fig:quadrant}). New Zealand achieves CRI = 29.3 despite high seismic exposure.

\begin{figure}[htbp]
\centerline{\includegraphics[width=\columnwidth]{figures/quadrant_matrix.png}}
\caption{Risk-Recovery Quadrant Matrix: Fighters (top-right) demonstrate high exposure with high recovery capacity. Bubble size encodes disaster deaths.}
\label{fig:quadrant}
\end{figure}

\textbf{Q2: Do wealthier nations recover faster, or does governance matter more?}
Governance dominates. Guyana (GDP \$23k, WGI = $-0.25$, CRI = 7.6) vs. Cabo Verde (GDP \$4k, WGI = $+0.58$, CRI = 19.7) demonstrates that wealth without governance is a fragile shield.

\textbf{Q3: How does resilience evolve alongside climate risk?}
The temporal view (Fig.~\ref{fig:governance_scatter}) reveals Europe widening its readiness-vulnerability gap (+0.07 improvement) while Africa stagnates (+0.02 over 20 years).

\begin{figure}[htbp]
\centerline{\includegraphics[width=\columnwidth]{figures/governance_wealth_scatter.png}}
\caption{Governance vs. Wealth scatter showing CRI stratification by governance tier at equivalent GDP levels---the governance premium visualized.}
\label{fig:governance_scatter}
\end{figure}

\section{Limitations and Future Work}

\subsection{Methodological Limitations}
\begin{enumerate}
    \item \textbf{Index Sensitivity}: Derived indices are sensitive to component weightings. For example, the DII uses a coefficient of 4 for affected population based on UNDRR literature \cite{b14}, but alternative weights (2--6) would shift country rankings. Sensitivity analysis showed top-quartile membership remains 78\% stable across weight variations.
    \item \textbf{Temporal Granularity}: Annual aggregation masks within-year recovery dynamics. Disasters occurring in December may show recovery effects only in subsequent year data, introducing measurement lag.
    \item \textbf{Causality}: Correlational analysis cannot establish causal direction. While governance correlates with resilience ($r = 0.78$), reverse causality---resilient institutions enabling good governance---cannot be ruled out without instrumental variable approaches.
    \item \textbf{Ecological Fallacy}: Country-level aggregation may mask sub-national variation. Within-country inequality in resilience (e.g., urban vs. rural) is not captured.
\end{enumerate}

\subsection{Data Limitations}
\begin{enumerate}
    \item \textbf{African Sparsity}: 18--19\% missing data for African nations 2000--2005
    \item \textbf{Gini Coverage}: Only 36.7\% coverage limits inequality analysis
    \item \textbf{INFORM Recency}: Data begins 2016, restricting historical hazard analysis
\end{enumerate}

\subsection{Future Directions}
\begin{enumerate}
    \item Real-time satellite telemetry integration (Sentinel, Landsat)
    \item Machine learning for resilience trajectory prediction
    \item Sub-national analysis using DesInventar granular data
    \item Treemap/Sankey visualizations for humanitarian aid flow analysis
    \item Causal inference methods (IV, DiD) for governance-resilience causality
\end{enumerate}

\section{Conclusion}
The Global Disaster Resilience Analytics Platform demonstrates that national resilience is fundamentally a function of governance, not merely economic wealth. By fusing 13 datasets into 4,584 country-year records with 102 features, and engineering three composite indices following the project's mathematical frameworks, we quantify the ``Resilience Paradox'': governance quality ($r = 0.78$) consistently outperforms GDP ($r = 0.66$) as a resilience predictor.

The interactive Tableau dashboard operationalizes these findings through coordinated views enabling exploration across time (2000--2023), geography (5 regions, 191 countries), disaster exposure levels, and socio-economic groups (governance tiers). Nations like Cabo Verde and Bhutan demonstrate that well-governed societies punch above their economic weight, while resource-rich nations with governance deficits remain structurally fragile.

\appendix

\section{Derived Attributes Reference}
Table~\ref{tab:derived} documents all derived attributes per project requirements (R2).

\begin{table}[htbp]
\caption{Derived Attributes with Computation Logic}
\label{tab:derived}
\centering
\footnotesize
\begin{tabular}{p{2.8cm}p{4.4cm}}
\toprule
\textbf{Attribute} & \textbf{Derivation Formula} \\
\midrule
fatalities\_per\_million & (emdat\_deaths $\times$ $10^6$) / population \\
affected\_pct & (emdat\_affected $\times$ 100) / population \\
gdp\_growth\_change & gdp\_growth[t] $-$ gdp\_growth[t-1] \\
infrastructure\_exposure & urban\_pct $\times$ inform\_hazard \\
normalized\_loss & (damage\_usd / gdp) $\times$ 100 \\
DII & (F + 4$\times$A) / GDP $\times$ S \\
RRS & ($\Delta$GDP + HDI + Gov) / R\_f \\
CRI & A\_c / (E + V + 0.001) \\
Resilience Quadrant & LOD median-based classification \\
Governance Tier & WGI threshold binning \\
\bottomrule
\end{tabular}
\end{table}

\section{Dataset Documentation}
Per project deliverable requirements, Table~\ref{tab:dataset_doc} provides the Dataset Documentation Sheet.

\begin{table}[htbp]
\caption{Dataset Documentation Sheet}
\label{tab:dataset_doc}
\centering
\footnotesize
\begin{tabular}{p{2cm}p{5.2cm}}
\toprule
\textbf{Field} & \textbf{Value} \\
\midrule
Dataset Name & unified\_resilience\_dataset.csv \\
Primary Key & (iso3, year) \\
Row Count & 4,584 (191 countries $\times$ 24 years) \\
Column Count & 102 \\
Year Range & 2000--2023 \\
Geographic Scope & 191 UN member states \\
Derived Indices & DII, RRS, CRI (normalized 0--100) \\
Missing Strategy & Within-country linear interpolation \\
\bottomrule
\end{tabular}
\end{table}

\section{Tableau Calculated Field Examples}

\textbf{Radar Chart X-Coordinate}:
\begin{lstlisting}
([Pivot Field Values] + 2.5) * 
  COS(RADIANS(MIN([Radar Angle])))
\end{lstlisting}

\textbf{Radar Chart Y-Coordinate}:
\begin{lstlisting}
([Pivot Field Values] + 2.5) * 
  SIN(RADIANS(MIN([Radar Angle])))
\end{lstlisting}

\textbf{Selection Count for Context Switching}:
\begin{lstlisting}
{FIXED : COUNTD(
  IF [Selected Countries Set] 
  THEN [iso3] END)}
\end{lstlisting}

\section{Validation and Quality Assurance}
The ETL pipeline includes automated validation checks:
\begin{itemize}
    \item \textbf{Coverage Verification}: All 102 columns validated for $>$25\% non-null values
    \item \textbf{Range Checks}: Normalized indices verified to [0, 100] bounds
    \item \textbf{Referential Integrity}: All ISO3 codes validated against pycountry library
    \item \textbf{Temporal Consistency}: Year-over-year changes flagged if $>$3 standard deviations
\end{itemize}

The validation report (\texttt{validation\_report.txt}) confirms:
\begin{itemize}
    \item DII Coverage: 99.4\%, Range: 0.0000--17.9065
    \item RRS Coverage: 100.0\%, Range: 0.2257--2.4044
    \item CRI Coverage: 100.0\%, Range: 0.0887--4.2822
\end{itemize}

\begin{thebibliography}{00}
\bibitem{b1} EM-DAT, ``The International Disaster Database,'' CRED, UCLouvain. [Online]. Available: https://www.emdat.be/
\bibitem{b2} World Bank, ``World Development Indicators,'' 2024. [Online]. Available: https://data.worldbank.org/
\bibitem{b3} D. Kaufmann, A. Kraay, and M. Mastruzzi, ``Worldwide Governance Indicators,'' World Bank Policy Research Working Paper No. 5430, 2010.
\bibitem{b4} Univ. of Notre Dame, ``ND-GAIN Country Index.'' [Online]. Available: https://gain.nd.edu/our-work/country-index/
\bibitem{b5} UNDP, ``Human Development Reports.'' [Online]. Available: https://hdr.undp.org/data-center
\bibitem{b6} GDACS, ``Global Disaster Alert and Coordination System.'' [Online]. Available: https://www.gdacs.org/
\bibitem{b7} IMF, ``World Economic Outlook Database.'' [Online]. Available: https://www.imf.org/en/Publications/WEO
\bibitem{b8} EC-JRC, ``INFORM Risk Index.'' [Online]. Available: https://drmkc.jrc.ec.europa.eu/inform-index
\bibitem{b9} J. Yao, ``Harmonized Global Nighttime Lights,'' 2019. [Online]. Available: https://figshare.com/articles/dataset/9828827
\bibitem{b10} OCHA, ``Financial Tracking Service.'' [Online]. Available: https://fts.unocha.org/
\bibitem{b11} UNDRR, ``DesInventar Sendai.'' [Online]. Available: https://www.desinventar.net/
\bibitem{b12} R. Barro and J. Lee, ``Educational Attainment Dataset.'' [Online]. Available: https://barrolee.github.io/BarroLeeDataSet/
\bibitem{b13} World Inequality Database. [Online]. Available: https://wid.world/data/
\bibitem{b14} UNDRR, ``Global Assessment Report on Disaster Risk Reduction,'' Geneva, 2022.
\bibitem{b15} Our World in Data, ``Natural Disasters.'' [Online]. Available: https://ourworldindata.org/natural-disasters
\bibitem{b16} Chen and Nordhaus, ``Global Gridded Geographically Based Economic Data (G-Econ),'' Yale University, 2011.
\end{thebibliography}

\end{document}
